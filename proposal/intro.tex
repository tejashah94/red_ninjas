\section{Introduction}\label{sec:intro}

With recent advances in computing technology, and new computing capabilities
with GPGPU hardware~\cite{owens2008gpu} and programming models~\cite{nvidia2007compute}, 
a trend of mapping many image processing applications on GPU is seen. 
Compute Unified Device Architecture (CUDA) has enabled mapping generic parallel
implementations of image processing algorithms~\cite{yang2008parallel} easier and benefits with good amount of 
performance and energy efficiency. 
Face detection is one such application which has lots of fine-grained
data parallelism available and exploit the execution resources of GPU.
It is processing of taking an image and detecting and locating the faces 
in the given image. It is an important application used world-wide for
public security, airports, video conferencing and video surveillance. 

Most of face detection systems today use cascade classifier algorithm 
based on Viola and Jones~\cite{viola2001rapid}. It has three important 
concepts tied to it -- integral image calculation, Adaboost classifier training
algorithm~\cite{freund1999short} and cascade classifier.  
Although, many of them have implemented these algorithm in CPUs, due to 
the inherent serial nature of CPU execution, you cannot get much of the
performance benefit and may not be able to meet hard real-time constraints,
even when executed on a multi-core CPU. 
With face detection algorithm's inherent parallel characteristics, GPGPU 
parallel computing substrate is a good candidate to gain performance benefits. 
With recent advances in NVIDIA CUDA programming model for scientific application 
acceleration~\cite{buck2007gpu}, we aim to use GPGPU execution model for accelerating face detection algorithm.

In this project, we intend to implement face detection algorithm 
based on the Viola Jones classifier on GPU. 
As a starting point, we take the GNU licensed C++ program that has the 
algorithm implemented to detect faces in images. 
There are various portions in the algorithm that 
can be parallelized and hence can leverage the execution of GPU resources efficiently. 
We plan to implement all three phases of the face detection -- integral image calculation,
scanning window stage and classifying the output from each classifying stage. 
As a course project we want to limit to the implementation of these 3 stages mentioned above, 
and not focus on the training of classifier itself. We take some insights and principles
based on previous implementations of face detection on GPGPUs, FPGA done here~\cite{kong2010gpu, sun2013acceleration, cho2009fpga}.
The focus of this project is
to gain performance benefits out of face detection acceleration and characterize
the bottlenecks if there are any. 

Section \ref{sec:algo} explains the Viola Jones algorithm briefly. Section \ref{sec:plan} explains the project phases and work distribution. Section \ref{sec:meth} explains the evaluation  methodology we consider for comparison and also explain the portions we are going to parallelize and offload to the GPU.


