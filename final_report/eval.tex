\section{Evaluation framework}\label{sec:eval}

\begin{table*}[th]
\centering
\scalebox{0.65}{
\begin{tabular}{|c|c|c|c|c|c|c|c|c|c|c|c|c|c|c|c|c|c|c|c|c|c|c|c|c|l|}
\hline
\textit{\textbf{Stage}}      & 1  & 2  & 3  & 4  & 5   & 6   & 7   & 8   & 9   & 10  & 11  & 12  & 13  & 14  & 15  & 16  & 17  & 18  & 19  & 20  & 21  & 22  & 23  & 24  & 25   \\ \hline 
\textit{\textbf{Classfiers}} & 9  & 16 & 27 & 32 & 52  & 53  & 62  & 72  & 83  & 91  & 99  & 115 & 127 & 135 & 136 & 137 & 159 & 155 & 169 & 196 & 197 & 181 & 199 & 211 & 200  \\ \hline
\textit{\textbf{Rectangles}} & 18 & 48 & 81 & 96 & 156 & 159 & 186 & 216 & 249 & 273 & 297 & 345 & 381 & 405 & 408 & 411 & 477 & 465 & 507 & 588 & 591 & 543 & 597 & 633 & 600  \\ \hline
\end{tabular}
}
\vspace{0.05in}
\caption{HAAR cascade classifier and its features}
\label{table:features}
\end{table*}

\begin{table*}[h]
    \centering
    \scalebox{0.75}{
    \begin{tabular}{|l|l|l|l|l|l|l|}
        \hline
        \textbf{Kernel}                   & \multicolumn{2}{l|}{\textbf{Registers}}    & \textbf{Shared memory (KB)} & \textbf{Constant Memory (Bytes)} & \multicolumn{2}{l|}{\textbf{Occupancy (\%)}} \\ \hline
        \textbf{}                         & \textbf{With Maxreg} & \textbf{w/o Maxreg} & \textbf{}                   & \textbf{}                        & \textbf{With Maxreg}  & \textbf{w/o Maxreg}  \\ \hline
        \textbf{NN + RowScan}             & 18                   & 18                  & 8.2                         & 88                               & 100                   & 100                  \\ \hline
        \textbf{Transpoce 1}              & 12                   & 12                  & 2.1                         & 60                               & 100                   & 100                  \\ \hline
        \textbf{RowScan Only}             & 17                   & 18                  & 8.2                         & 72                               & 100                   & 66.67                \\ \hline
        \textbf{Transpose 2}              & 12                   & 12                  & 2.1                         & 60                               & 100                   & 100                  \\ \hline
        \textbf{HAAR Cascade Classifiers} & 20                   & 28                  & 19.5                        & 156                              & 66.67                 & 66.67                \\ \hline
    \end{tabular}
    }
    \vspace{0.05in}
    \caption{Registers, Shared Memory, Constant Memory and Occupancy for all the kernels}
    \label{table:util}
\end{table*}




For this project, we evaluate the performance and resource utilization 
of face detecion algorithm on GPGPU implementation with that of a CPU. 
We are not considering the
cascade classifier training part and are directly taking a CPU version of
previosuly trained classifier. Offline training of the classifier network
for different images on GPU will be implemented as part of our future work.
For now, the trained cascade classifier consists of 
number of stages needed for face detection, HAAR features needed in each
stage, the rectangles of each feeature, threshold values for each stage and classifier.
Table~\ref{table:features} gives an overview and information of the total number os stages
used, number of filers/classifiers in each stage and rectangle features used in each stage.
Apart form that, each rectangle has a weight associated with it, each classifier has a threshold value and each stage
also has a threshold value. These threshold values are used at each step of scan window processing
and would determine whether the face is present in the downsampled image. 

\begin{table}[h]
    \centering
    \scalebox{0.85}{
    \begin{tabular}{|l|l|}
        \hline
        \textbf{CUDA runtime version}         & 7000                  \\ \hline
        \textbf{CUDA driver version}          & 7050                  \\ \hline
        \textbf{}                             &                       \\ \hline
        \textbf{Device Name}                  & GeForce GTX 480       \\ \hline
        \textbf{Compute Capability}           & 2.0                   \\ \hline
        \textbf{Global Memory (MB}            & 1535                  \\ \hline
        \textbf{Total Const Memory (KB}       & 64                    \\ \hline
        \textbf{Shared Memory per Block (KB)} & 48                    \\ \hline
        \textbf{Shared Memory per SM (KB)}    & 48                    \\ \hline
        \textbf{L2 Cache Size (KB)}           & 768                   \\ \hline
        \textbf{Registers per Block}          & 32768                 \\ \hline
        \textbf{Registers per SM}             & 32768                 \\ \hline
        \textbf{SM Count}                     & 15                    \\ \hline
        \textbf{max threads per SM}           & 1536                  \\ \hline
        \textbf{Max threads per block}        & 1024                  \\ \hline
        \textbf{Max thread dims}              & (1024, 1024, 64)      \\ \hline
        \textbf{Max grid size}                & (65535, 65535, 65535) \\ \hline
        \textbf{Num copy engines}             & 1                     \\ \hline
    \end{tabular}
    }
    \vspace{0.05in}
    \caption{Arch. details of the GPU card used}
    \label{table:device}
\end{table}

\vspace{-0.1in}
\begin{table}[h]
    \centering
    \scalebox{0.85}{
    \begin{tabular}{|l|l|}
        \hline
        \textbf{Architecture}        & x86\_64                            \\ \hline
        \textbf{CPU(s)}              & 16                                 \\ \hline
        \textbf{On-line CPU(s) list} & 0-15                               \\ \hline
        \textbf{Thread(s) per core}  & 2                                  \\ \hline
        \textbf{Core(s) per socket}  & 4                                  \\ \hline
        \textbf{Socket(s)}           & 2                                  \\ \hline
        \textbf{Model name:}         & Intel Xeon(R) CPU E5520 @ 2.27 GHz \\ \hline
        \textbf{CPU MHz}             & 1600                               \\ \hline
        \textbf{L1d cache (KB)}      & 32                                 \\ \hline
        \textbf{L1i cache (32KB)}    & 32                                 \\ \hline
        \textbf{L2 cache (KB)}       & 256                                \\ \hline
        \textbf{L3 cache (MB)}       & 8                                  \\ \hline
    \end{tabular}
    }
    \vspace{0.05in}
    \caption{Arch. details of the CPU used}
    \label{table:cpu}
\end{table}

\paragraph{}
We compared the GPU exclsuive time (kernel time),
inclsuive time (kernel + CPU to GPU copy time) with the CPU execution time 
for face detection in an image.
As mentioned in Section~\ref{sec:impl}, we have parallelized three different
portions in the algorithm and in Section~\ref{sec:results} we analyze the results of each of this kernel.
We evaluated our face detection implementation on an NVIDIA GTX 480 GPU card with 15 SMs and 1.5GB global memory.
Table~\ref{table:device} gives the architecture details of the device on which we evaluated the kernels. For comparison with CPU, 
we used 16 core Intel Xeon CPU, but since the Viola Jones code was a single threaded code, the comparision is only with
1 core of CPU. Table~\ref{table:cpu} gives the architecture details of the CPU used for comaprison. 
We also use all the optimizations applied for the kernels (Section~\ref{sec:optim}) and show the
results for these optimizations. 

For profiling and bottleneck analysis, we used
NVIDIA Visual Profiler~\cite{rosen2013visual} and then take corresponding de-
cisions for performance boost. CUDA profilers also helped us to gain an understadning 
of the kernel occupancy and thier shared memory usage.
We use CUDA events~\cite{wilt2013cuda} for timing analysis of GPGPU (both exclusive and inclusive) and
CPU execution.  


